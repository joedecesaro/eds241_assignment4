% Options for packages loaded elsewhere
\PassOptionsToPackage{unicode}{hyperref}
\PassOptionsToPackage{hyphens}{url}
%
\documentclass[
]{article}
\usepackage{amsmath,amssymb}
\usepackage{lmodern}
\usepackage{ifxetex,ifluatex}
\ifnum 0\ifxetex 1\fi\ifluatex 1\fi=0 % if pdftex
  \usepackage[T1]{fontenc}
  \usepackage[utf8]{inputenc}
  \usepackage{textcomp} % provide euro and other symbols
\else % if luatex or xetex
  \usepackage{unicode-math}
  \defaultfontfeatures{Scale=MatchLowercase}
  \defaultfontfeatures[\rmfamily]{Ligatures=TeX,Scale=1}
\fi
% Use upquote if available, for straight quotes in verbatim environments
\IfFileExists{upquote.sty}{\usepackage{upquote}}{}
\IfFileExists{microtype.sty}{% use microtype if available
  \usepackage[]{microtype}
  \UseMicrotypeSet[protrusion]{basicmath} % disable protrusion for tt fonts
}{}
\makeatletter
\@ifundefined{KOMAClassName}{% if non-KOMA class
  \IfFileExists{parskip.sty}{%
    \usepackage{parskip}
  }{% else
    \setlength{\parindent}{0pt}
    \setlength{\parskip}{6pt plus 2pt minus 1pt}}
}{% if KOMA class
  \KOMAoptions{parskip=half}}
\makeatother
\usepackage{xcolor}
\IfFileExists{xurl.sty}{\usepackage{xurl}}{} % add URL line breaks if available
\IfFileExists{bookmark.sty}{\usepackage{bookmark}}{\usepackage{hyperref}}
\hypersetup{
  pdftitle={EDS241: Assignment 4},
  pdfauthor={Joe DeCesaro},
  hidelinks,
  pdfcreator={LaTeX via pandoc}}
\urlstyle{same} % disable monospaced font for URLs
\usepackage[margin=1in]{geometry}
\usepackage{color}
\usepackage{fancyvrb}
\newcommand{\VerbBar}{|}
\newcommand{\VERB}{\Verb[commandchars=\\\{\}]}
\DefineVerbatimEnvironment{Highlighting}{Verbatim}{commandchars=\\\{\}}
% Add ',fontsize=\small' for more characters per line
\usepackage{framed}
\definecolor{shadecolor}{RGB}{248,248,248}
\newenvironment{Shaded}{\begin{snugshade}}{\end{snugshade}}
\newcommand{\AlertTok}[1]{\textcolor[rgb]{0.94,0.16,0.16}{#1}}
\newcommand{\AnnotationTok}[1]{\textcolor[rgb]{0.56,0.35,0.01}{\textbf{\textit{#1}}}}
\newcommand{\AttributeTok}[1]{\textcolor[rgb]{0.77,0.63,0.00}{#1}}
\newcommand{\BaseNTok}[1]{\textcolor[rgb]{0.00,0.00,0.81}{#1}}
\newcommand{\BuiltInTok}[1]{#1}
\newcommand{\CharTok}[1]{\textcolor[rgb]{0.31,0.60,0.02}{#1}}
\newcommand{\CommentTok}[1]{\textcolor[rgb]{0.56,0.35,0.01}{\textit{#1}}}
\newcommand{\CommentVarTok}[1]{\textcolor[rgb]{0.56,0.35,0.01}{\textbf{\textit{#1}}}}
\newcommand{\ConstantTok}[1]{\textcolor[rgb]{0.00,0.00,0.00}{#1}}
\newcommand{\ControlFlowTok}[1]{\textcolor[rgb]{0.13,0.29,0.53}{\textbf{#1}}}
\newcommand{\DataTypeTok}[1]{\textcolor[rgb]{0.13,0.29,0.53}{#1}}
\newcommand{\DecValTok}[1]{\textcolor[rgb]{0.00,0.00,0.81}{#1}}
\newcommand{\DocumentationTok}[1]{\textcolor[rgb]{0.56,0.35,0.01}{\textbf{\textit{#1}}}}
\newcommand{\ErrorTok}[1]{\textcolor[rgb]{0.64,0.00,0.00}{\textbf{#1}}}
\newcommand{\ExtensionTok}[1]{#1}
\newcommand{\FloatTok}[1]{\textcolor[rgb]{0.00,0.00,0.81}{#1}}
\newcommand{\FunctionTok}[1]{\textcolor[rgb]{0.00,0.00,0.00}{#1}}
\newcommand{\ImportTok}[1]{#1}
\newcommand{\InformationTok}[1]{\textcolor[rgb]{0.56,0.35,0.01}{\textbf{\textit{#1}}}}
\newcommand{\KeywordTok}[1]{\textcolor[rgb]{0.13,0.29,0.53}{\textbf{#1}}}
\newcommand{\NormalTok}[1]{#1}
\newcommand{\OperatorTok}[1]{\textcolor[rgb]{0.81,0.36,0.00}{\textbf{#1}}}
\newcommand{\OtherTok}[1]{\textcolor[rgb]{0.56,0.35,0.01}{#1}}
\newcommand{\PreprocessorTok}[1]{\textcolor[rgb]{0.56,0.35,0.01}{\textit{#1}}}
\newcommand{\RegionMarkerTok}[1]{#1}
\newcommand{\SpecialCharTok}[1]{\textcolor[rgb]{0.00,0.00,0.00}{#1}}
\newcommand{\SpecialStringTok}[1]{\textcolor[rgb]{0.31,0.60,0.02}{#1}}
\newcommand{\StringTok}[1]{\textcolor[rgb]{0.31,0.60,0.02}{#1}}
\newcommand{\VariableTok}[1]{\textcolor[rgb]{0.00,0.00,0.00}{#1}}
\newcommand{\VerbatimStringTok}[1]{\textcolor[rgb]{0.31,0.60,0.02}{#1}}
\newcommand{\WarningTok}[1]{\textcolor[rgb]{0.56,0.35,0.01}{\textbf{\textit{#1}}}}
\usepackage{longtable,booktabs,array}
\usepackage{calc} % for calculating minipage widths
% Correct order of tables after \paragraph or \subparagraph
\usepackage{etoolbox}
\makeatletter
\patchcmd\longtable{\par}{\if@noskipsec\mbox{}\fi\par}{}{}
\makeatother
% Allow footnotes in longtable head/foot
\IfFileExists{footnotehyper.sty}{\usepackage{footnotehyper}}{\usepackage{footnote}}
\makesavenoteenv{longtable}
\usepackage{graphicx}
\makeatletter
\def\maxwidth{\ifdim\Gin@nat@width>\linewidth\linewidth\else\Gin@nat@width\fi}
\def\maxheight{\ifdim\Gin@nat@height>\textheight\textheight\else\Gin@nat@height\fi}
\makeatother
% Scale images if necessary, so that they will not overflow the page
% margins by default, and it is still possible to overwrite the defaults
% using explicit options in \includegraphics[width, height, ...]{}
\setkeys{Gin}{width=\maxwidth,height=\maxheight,keepaspectratio}
% Set default figure placement to htbp
\makeatletter
\def\fps@figure{htbp}
\makeatother
\setlength{\emergencystretch}{3em} % prevent overfull lines
\providecommand{\tightlist}{%
  \setlength{\itemsep}{0pt}\setlength{\parskip}{0pt}}
\setcounter{secnumdepth}{5}
\setlength{\parindent}{1em}
\usepackage{float}
\ifluatex
  \usepackage{selnolig}  % disable illegal ligatures
\fi

\title{EDS241: Assignment 4}
\author{Joe DeCesaro}
\date{03/08/2022}

\begin{document}
\maketitle

\hypertarget{homework-4}{%
\section{Homework 4}\label{homework-4}}

\noindent This question will ask you to estimate the price elasticity of
demand for fresh sardines across 56 ports located in 4 European
countries with monthly data from 2013 to 2019. The data are contained in
the file EU\_sardines.csv, which is available on Gauchospace.

\noindent Each row in the data file is a combination of port location
(where the fish is landed and sold) in a given year and month. You can
ignore the fact that the sample is not balanced (the number of monthly
observations varies across ports).

\noindent For the assignment, you will need the following variables:
year, month, country, port (port where sardines are landed and sold),
price\_euro\_kg (price per kg in €), and volume\_sold\_kg (quantity of
sardines sold in kg). In the questions below, I use log() to denote the
natural logarithm.

\hypertarget{read-in-the-data}{%
\subsection{Read in the data}\label{read-in-the-data}}

\begin{Shaded}
\begin{Highlighting}[]
\NormalTok{fish\_data }\OtherTok{\textless{}{-}}\FunctionTok{read.csv}\NormalTok{(}\FunctionTok{here}\NormalTok{(}\StringTok{"data/EU\_sardines.csv"}\NormalTok{))}
\end{Highlighting}
\end{Shaded}

\hypertarget{a-estimate-a-bivariate-regression-of-logvolume_sold_kg-on-logprice_euro_kg.-what-is-the-price-elasticity-of-demand-for-sardines-test-the-null-hypothesis-that-the-price-elasticity-is-equal-to--1.}{%
\subsection{(a) Estimate a bivariate regression of log(volume\_sold\_kg)
on log(price\_euro\_kg). What is the price elasticity of demand for
sardines? Test the null hypothesis that the price elasticity is equal to
-1.}\label{a-estimate-a-bivariate-regression-of-logvolume_sold_kg-on-logprice_euro_kg.-what-is-the-price-elasticity-of-demand-for-sardines-test-the-null-hypothesis-that-the-price-elasticity-is-equal-to--1.}}

\begin{Shaded}
\begin{Highlighting}[]
\NormalTok{fish\_data }\OtherTok{\textless{}{-}}\NormalTok{ fish\_data }\SpecialCharTok{\%\textgreater{}\%} 
  \FunctionTok{mutate}\NormalTok{(}\AttributeTok{log\_sold =} \FunctionTok{log}\NormalTok{(volume\_sold\_kg),}
         \AttributeTok{log\_price =} \FunctionTok{log}\NormalTok{(price\_euro\_kg))}
\end{Highlighting}
\end{Shaded}

\begin{Shaded}
\begin{Highlighting}[]
\NormalTok{model1 }\OtherTok{\textless{}{-}} \FunctionTok{lm\_robust}\NormalTok{(}\AttributeTok{formula =}\NormalTok{ log\_sold }\SpecialCharTok{\textasciitilde{}}\NormalTok{ log\_price,}
             \AttributeTok{data =}\NormalTok{ fish\_data)}
\end{Highlighting}
\end{Shaded}

\begin{Shaded}
\begin{Highlighting}[]
\NormalTok{model1\_table }\OtherTok{\textless{}{-}} \FunctionTok{tidy}\NormalTok{(model1)}
\NormalTok{model1\_table }\SpecialCharTok{\%\textgreater{}\%}
  \FunctionTok{select}\NormalTok{(term, estimate, std.error, p.value, conf.low, conf.high) }\SpecialCharTok{\%\textgreater{}\%}
  \FunctionTok{kable}\NormalTok{()}
\end{Highlighting}
\end{Shaded}

\begin{longtable}[]{@{}lrrrrr@{}}
\toprule
term & estimate & std.error & p.value & conf.low & conf.high \\
\midrule
\endhead
(Intercept) & 7.759061 & 0.0430246 & 0 & 7.674709 & 7.843413 \\
log\_price & -1.545335 & 0.0781254 & 0 & -1.698505 & -1.392166 \\
\bottomrule
\end{longtable}

\noident The estimated price elasticity of demand for sardines
regression log\_sold on log\_price is -1.55. We can say with 95\%
confidence that the interval {[}-1.7, -1.39{]} contains the true
\(\beta_1\). Because this CI does not contain -1 we can the null
hypothesis that the price elasticity is equal to -1.

\hypertarget{b-like-in-lecture-8-see-the-iv.r-script-we-will-use-wind_m_s-as-an-instrument-for-logprice_euro_kg.-to-begin-estimate-the-first-stage-regression-relating-logprice_euro_kg-to-wind_m_s.-interpret-the-estimated-coefficient-on-wind-speed.-does-it-have-the-expected-sign-also-test-for-the-relevance-of-the-instrument-and-whether-it-is-a-weak-instrument-by-reporting-the-proper-f-statistic.}{%
\subsection{(b) Like in Lecture 8 (see the IV.R script), we will use
wind\_m\_s as an instrument for log(price\_euro\_kg). To begin, estimate
the first-stage regression relating log(price\_euro\_kg) to wind\_m\_s.
Interpret the estimated coefficient on wind speed. Does it have the
expected sign? Also test for the relevance of the instrument and whether
it is a ``weak'' instrument by reporting the proper
F-statistic.}\label{b-like-in-lecture-8-see-the-iv.r-script-we-will-use-wind_m_s-as-an-instrument-for-logprice_euro_kg.-to-begin-estimate-the-first-stage-regression-relating-logprice_euro_kg-to-wind_m_s.-interpret-the-estimated-coefficient-on-wind-speed.-does-it-have-the-expected-sign-also-test-for-the-relevance-of-the-instrument-and-whether-it-is-a-weak-instrument-by-reporting-the-proper-f-statistic.}}

\begin{Shaded}
\begin{Highlighting}[]
\CommentTok{\# fs1 \textless{}{-} lm\_robust(formula = log\_price \textasciitilde{} wind\_m\_s,}
\CommentTok{\#           data = fish\_data)}
\CommentTok{\# }
\CommentTok{\# hux\_fs1 \textless{}{-} huxreg(fs1)}
\CommentTok{\# hux\_fs1}
\end{Highlighting}
\end{Shaded}


\end{document}
